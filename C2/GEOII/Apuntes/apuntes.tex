%%%%%%%%%%%%%%%%%%%%%%%%%%%%%%%%%%%%%%%%%%%%%%%%%%%%%%%%%%%%%%%%%%%%%%%%%
%
% Plantilla para libro de texto de matemáticas.
%
% Esta plantilla ha sido desarrollada desde cero, pero utiliza algunas partes
% del código de la plantilla original utilizada en apuntesDGIIM
% (https://github.com/libreim/apuntesDGIIM), basada a su vez en las plantillas
% 'Short Sectioned Assignment' de Frits Wenneker (http://www.howtotex.com),
% 'Plantilla de Trabajo' de Mario Román y 'Plantilla básica de Latex en Español'
% de Andrés Herrera Poyatos (https://github.com/andreshp). También recoge
% ideas de la plantilla 'Multi-Purpose Large Font Title Page' de
% Frits Wenneker y Vel (vel@latextemplates.com).
%
% Licencia:
% CC BY-NC-SA 4.0 (https://creativecommons.org/licenses/by-nc-sa/4.0/)
%
%%%%%%%%%%%%%%%%%%%%%%%%%%%%%%%%%%%%%%%%%%%%%%%%%%%%%%%%%%%%%%%%%%%%%%%%%

% ---------------------------------------------------------------------------
% CONFIGURACIÓN BÁSICA DEL DOCUMENTO
% ---------------------------------------------------------------------------

%\documentclass[11pt, a4paper, twoside]{article} % Usar para imprimir
\documentclass[10pt, a4paper]{article}

\linespread{1.3}            % Espaciado entre líneas.
\setlength\parindent{0pt}   % No indentar el texto por defecto.
\setlength\parskip{7pt}

% ---------------------------------------------------------------------------
% PAQUETES BÁSICOS
% ---------------------------------------------------------------------------

% IDIOMA
\usepackage[utf8]{inputenc}
\usepackage[spanish, es-tabla, es-lcroman, es-noquoting]{babel}

% MATEMÁTICAS
\usepackage{amsmath}    % Paquete básico de matemáticas
\usepackage{amsthm}     % Teoremas
\usepackage{mathrsfs}   % Fuente para ciertas letras utilizadas en matemáticas

% FUENTES
\usepackage{newpxtext, newpxmath}   % Fuente similar a Palatino
\usepackage{FiraSans}                 % Fuente sans serif
\usepackage[T1]{fontenc}
\usepackage[italic]{mathastext}     % Utiliza la fuente del documento
                                    % en los entornos matemáticos

% MÁRGENES
\usepackage[margin=2.5cm, top=3cm]{geometry}

% LISTAS
\usepackage{enumitem}       % Mejores listas
\setlist{leftmargin=.5in}   % Especifica la indentación para las listas.

% Listas ordenadas con números romanos (i), (ii), etc.
\newenvironment{nlist}
{\begin{enumerate}
    \renewcommand\labelenumi{(\emph{\roman{enumi})}}}
  {\end{enumerate}}

%  OTROS
\usepackage{hyperref}   % Enlaces
\usepackage{graphicx}   % Permite incluir gráficos en el documento


% ---------------------------------------------------------------------------
% COLORES
% ---------------------------------------------------------------------------

\usepackage{xcolor}     % Permite definir y utilizar colores

\definecolor{50}{HTML}{E0F2F1}
\definecolor{100}{HTML}{B2DFDB}
\definecolor{200}{HTML}{80CBC4}
\definecolor{300}{HTML}{4DB6AC}
\definecolor{400}{HTML}{26A69A}
\definecolor{500}{HTML}{009688}
\definecolor{600}{HTML}{00897B}
\definecolor{700}{HTML}{00796B}
\definecolor{800}{HTML}{00695C}
\definecolor{900}{HTML}{004D40}

% ---------------------------------------------------------------------------
% DISEÑO DE PÁGINA
% ---------------------------------------------------------------------------

\usepackage{pagecolor}
\usepackage{afterpage}

% ---------------------------------------------------------------------------
% CABECERA Y PIE DE PÁGINA
% ---------------------------------------------------------------------------

\usepackage{fancyhdr}   % Paquete para cabeceras y pies de página

% Indica que las páginas usarán la configuración de fancyhdr
\pagestyle{fancy}
\fancyhf{}

% Representa la sección de la cabecera
\renewcommand{\sectionmark}[1]{%
\markboth{#1}{}}

% Representa la subsección de la cabecera
\renewcommand{\subsectionmark}[1]{%
\markright{#1}{}}

% Parte derecha de la cabecera
\fancyhead[LE,RO]{\sffamily\textsl{\rightmark} \hspace{1em}  \textcolor{500}{\rule[-0.4ex]{0.2ex}{1.2em}} \hspace{1em} \thepage}

% Parte izquierda de la cabecera
\fancyhead[RE,LO]{\sffamily{\leftmark}}

% Elimina la línea de la cabecera
\renewcommand{\headrulewidth}{0pt}

% Controla la altura de la cabecera para que no haya errores
\setlength{\headheight}{14pt}

% ---------------------------------------------------------------------------
% TÍTULOS DE PARTES Y SECCIONES
% ---------------------------------------------------------------------------

\usepackage{titlesec}

% Estilo de los títulos de las partes
\titleformat{\part}[hang]{\Huge\bfseries\sffamily}{\thepart\hspace{20pt}\textcolor{500}{|}\hspace{20pt}}{0pt}{\Huge\bfseries}
\titlespacing*{\part}{0cm}{-2em}{2em}[0pt]

% Reiniciamos el contador de secciones entre partes (opcional)
\makeatletter
\@addtoreset{section}{part}
\makeatother

% Estilo de los títulos de las secciones, subsecciones y subsubsecciones
\titleformat{\section}
  {\Large\bfseries\sffamily}{\thesection}{1em}{}

\titleformat{\subsection}
  {\Large\sffamily}{\thesubsection}{1em}{}[\vspace{.5em}]

\titleformat{\subsubsection}
  {\sffamily}{\thesubsubsection}{1em}{}

% ---------------------------------------------------------------------------
% ENTORNOS PERSONALIZADOS
% ---------------------------------------------------------------------------

\usepackage{mdframed}

%% DEFINICIONES DE LOS ESTILOS

% Nuevo estilo para definiciones
\newtheoremstyle{definition-style}  % Nombre del estilo
{}                                  % Espacio por encima
{}                                  % Espacio por debajo
{}                                  % Fuente del cuerpo
{}                                  % Identación
{\bf\sffamily}                      % Fuente para la cabecera
{.}                                 % Puntuación tras la cabecera
{.5em}                              % Espacio tras la cabecera
{\thmname{#1}\thmnumber{ #2}\thmnote{ (#3)}}  % Especificación de la cabecera

% Nuevo estilo para notas
\newtheoremstyle{remark-style}
{10pt}
{10pt}
{}
{}
{\itshape \sffamily}
{.}
{.5em}
{}

% Nuevo estilo para teoremas y proposiciones
\newtheoremstyle{theorem-style}
{}
{}
{}
{}
{\bfseries \sffamily}
{.}
{.5em}
{\thmname{#1}\thmnumber{ #2}\thmnote{ (#3)}}

% Nuevo estilo para ejemplos
\newtheoremstyle{example-style}
{10pt}
{10pt}
{}
{}
{\bf \sffamily}
{}
{.5em}
{\thmname{#1}\thmnumber{ #2.}\thmnote{ #3.}}

% Nuevo estilo para la demostración

\makeatletter
\renewenvironment{proof}[1][\proofname] {\par\pushQED{\qed}\normalfont\topsep6\p@\@plus6\p@\relax\trivlist\item[\hskip\labelsep\itshape\sffamily#1\@addpunct{.}]\ignorespaces}{\popQED\endtrivlist\@endpefalse}
\makeatother

%% ASIGNACIÓN DE LOS ESTILOS

% Teoremas, proposiciones y corolarios
\theoremstyle{theorem-style}
\newtheorem{nth}{Teorema}[section]
\newtheorem{nprop}{Proposición}[section]
\newtheorem{ncor}{Corolario}[section]
\newtheorem{lema}{Lema}[section]

% Definiciones
\theoremstyle{definition-style}
\newtheorem{ndef}{Definición}[section]

% Notas
\theoremstyle{remark-style}
\newtheorem*{nota}{Nota}

% Ejemplos
\theoremstyle{example-style}
\newtheorem{ejemplo}{Ejemplo}[section]

% Ejercicios y solución
\theoremstyle{definition-style}
\newtheorem{ejer}{Ejercicio}[section]

\theoremstyle{remark-style}
\newtheorem*{sol}{Solución}

%% MARCOS DE LOS ESTILOS

% Configuración general de mdframe, los estilos de los teoremas, etc
\mdfsetup{
  skipabove=1em,
  skipbelow=1em,
  innertopmargin=1em,
  innerbottommargin=1em,
  splittopskip=2\topsep,
}

% Definimos los marcos de los estilos

\mdfdefinestyle{nth-frame}{
	linewidth=2pt, %
	linecolor= 500, %
	topline=false, %
	bottomline=false, %
	rightline=false,%
	leftmargin=0em, %
	innerleftmargin=1em, %
  innerrightmargin=1em,
	rightmargin=0em, %
}%

\mdfdefinestyle{nprop-frame}{
	linewidth=2pt, %
	linecolor= 300, %
	topline=false, %
	bottomline=false, %
	rightline=false,%
	leftmargin=0pt, %
	innerleftmargin=1em, %
	innerrightmargin=1em,
	rightmargin=0pt, %
}%

\mdfdefinestyle{ndef-frame}{
	linewidth=2pt, %
	linecolor= 500, %
	backgroundcolor= 50,
	topline=false, %
	bottomline=false, %
	rightline=false,%
	leftmargin=0pt, %
	innerleftmargin=1em, %
	innerrightmargin=1em,
	rightmargin=0pt, %
}%

\mdfdefinestyle{ejer-frame}{
	linewidth=2pt, %
	linecolor= 300, %
	backgroundcolor= 50,
	topline=false, %
	bottomline=false, %
	rightline=false,%
	leftmargin=0pt, %
	innerleftmargin=1em, %
	innerrightmargin=1em,
	rightmargin=0pt, %
}%

\mdfdefinestyle{ejemplo-frame}{
	linewidth=0pt, %
	linecolor= 300, %
	leftline=false, %
	rightline=false, %
	leftmargin=0pt, %
	innerleftmargin=1.3em, %
	innerrightmargin=1em,
	rightmargin=0pt, %
	innertopmargin=0em,%
	innerbottommargin=0em, %
	splittopskip=\topskip, %
}%

% Asignamos los marcos a los estilos
\surroundwithmdframed[style=nth-frame]{nth}
\surroundwithmdframed[style=nprop-frame]{nprop}
\surroundwithmdframed[style=nprop-frame]{ncor}
\surroundwithmdframed[style=ndef-frame]{ndef}
\surroundwithmdframed[style=ejer-frame]{ejer}
\surroundwithmdframed[style=ejemplo-frame]{ejemplo}
\surroundwithmdframed[style=ejemplo-frame]{sol}

% ---------------------------------------------------------------------------
% CONFIGURACIÓN DE LA PORTADA
% ---------------------------------------------------------------------------

\newcommand{\asignatura}{
  Geometría II
}

\newcommand{\autor}{DGIIMUnderground · LibreIM}

\newcommand{\grado}{1º Doble Grado en Ingeniería Informática y Matemáticas}

\newcommand{\universidad}{Universidad de Granada}

\newcommand{\enlaceweb}{github.com/DGIIMUnderground}

% ---------------------------------------------------------------------------
% CONFIGURACIÓN PERSONALIZADA
% ---------------------------------------------------------------------------

%%%%%%%%%%%%%%%%%%%%%%%%%%%%%%%%%%%%%%%%%%%%%%%%%%%%%%%%%%%%%%%%%%%%%%%%%%%%%
% ---------------------------------------------------------------------------
% COMIENZO DEL DOCUMENTO
% ---------------------------------------------------------------------------
%%%%%%%%%%%%%%%%%%%%%%%%%%%%%%%%%%%%%%%%%%%%%%%%%%%%%%%%%%%%%%%%%%%%%%%%%%%%%

\begin{document}

% ---------------------------------------------------------------------------
% PORTADA EXTERIOR
% ---------------------------------------------------------------------------

\newpagecolor{500}\afterpage{\restorepagecolor} % Color de la página
\begin{titlepage}

  % Título del documento
	\parbox[t]{\textwidth}{
			\raggedright % Texto alineado a la izquierda
			\fontsize{50pt}{50pt}\selectfont\sffamily\color{white}{
			  \textbf{\asignatura}
      }
	}

	\vfill

	%% Autor e información del documento
	\parbox[t]{\textwidth}{
		\raggedright % Texto alineado a la izquierda
		\sffamily\large\color{white}
		{\Large \autor }\\[4pt]
		\grado\\
		\universidad\\[4pt]
		\texttt{\enlaceweb}
	}

\end{titlepage}

% ---------------------------------------------------------------------------
% PÁGINA DE LICENCIA
% ---------------------------------------------------------------------------

\thispagestyle{empty}
\null
\vfill

%% Información sobre la licencia
\parbox[t]{\textwidth}{
  \includegraphics{by-nc-sa.pdf}\\[4pt]
  \raggedright % Texto alineado a la izquierda
  \sffamily\large
  {\Large Este libro se distribuye bajo una licencia CC BY-NC-SA 4.0.}\\[4pt]
  Eres libre de distribuir y adaptar el material siempre que reconozcas a los\\
  autores originales del documento, no lo utilices para fines comerciales\\
  y lo distribuyas bajo la misma licencia.\\[4pt]
  \texttt{creativecommons.org/licenses/by-nc-sa/4.0/}
}

% ---------------------------------------------------------------------------
% PORTADA INTERIOR
% ---------------------------------------------------------------------------

\begin{titlepage}

  % Título del documento
	\parbox[t]{\textwidth}{
			\raggedright % Texto alineado a la izquierda
			\fontsize{50pt}{50pt}\selectfont\sffamily\color{500}{
			  \textbf{\asignatura}
      }
	}

	\vfill

	%% Autor e información del documento
	\parbox[t]{\textwidth}{
		\raggedright % Texto alineado a la izquierda
		\sffamily\large
		{\Large \autor}\\[4pt]
		\grado\\
		\universidad\\[4pt]
		\texttt{\enlaceweb}
	}

\end{titlepage}

% ---------------------------------------------------------------------------
% ÍNDICE
% ---------------------------------------------------------------------------

\thispagestyle{empty}
\tableofcontents
\newpage

% ---------------------------------------------------------------------------
% CONTENIDO
% ---------------------------------------------------------------------------

\part{Teoría}

\section*{Introducción}

Lorem ipsum dolor sit amet, consectetur adipiscing elit. Ut a nisi id mi dapibus commodo. Nam ac libero ultrices, posuere erat eu, tempor dolor. Curabitur porttitor, nulla at consequat mollis, mi turpis varius elit, vitae eleifend turpis est vel sem. Sed ut vehicula quam. Praesent id sem sed sapien tincidunt iaculis at quis dolor. Integer et magna quis sapien elementum pharetra. Pellentesque porttitor dapibus nibh, eget ullamcorper risus eleifend vitae. Nulla ac nulla nec orci scelerisque eleifend id sit amet arcu. Vivamus sodales, nibh in aliquet fringilla, quam erat tristique diam, quis dignissim justo mauris et lorem.

\pagebreak

\section{Formas bilineales y formas cuadráticas}

Como introducción, tomaremos el \textbf{producto escalar}:

\hspace{1cm} $ \rm I\!R^3 \text{ espacio vectoral sobre un cuerpo } K, \hspace{0.5cm} x=(x_1,x_2,x_3), y=(y_1,y_2,y_3) \in \rm I\!R^3$ \\
$$\cdot : \rm I\!R^3 \rightarrow \rm I\!R, \hspace{0.5cm} (x,y) \mapsto x \cdot y = x_1y_1+x_2y_2+x_3y_3=\sum^3_{i=1}x_iy_i$$
\subsubsection*{Propiedades del producto escalar}

\begin{enumerate}
	\item $ (x+x')\cdot y=x \cdot y+x' \cdot y \text{\hspace{0.5cm}y\hspace{0.5cm}} x\cdot(y+y') = x \cdot y + x\cdot y' \hspace{0.5cm} \forall x, y, x', y' \in \rm I\!R^3$
	\item $ (\alpha x)\cdot y = \alpha x\cdot y \text{\hspace{0.5cm}y\hspace{0.5cm}} x\cdot(\alpha y) = \alpha x\cdot y \hspace{0.5cm} \forall x, y \in \rm I\!R ^3, \hspace{0.2cm} \forall \alpha \in K $
	\item $ x\cdot y = y\cdot x \hspace{0.5cm} \forall x, y \in \rm I\!R ^3 $
	\item $ x\cdot x \geq 0 \text{\hspace{0.5cm}y\hspace{0.5cm}} x\cdot x = 0 \iff x=0 \hspace{0.5cm} \forall x \in \rm I\!R ^3$
\end{enumerate}

%\begin{align*}
%          \sin A \cos B &= \frac{1}{2}\left[ \sin(A-B)+\sin(A+B) \right] \\
%         \sin A \sin B &= \frac{1}{2}\left[ \sin(A-B)-\cos(A+B) \right] \\
%         \cos A \cos B &= \frac{1}{2}\left[ \cos(A-B)+\cos(A+B) \right] \\
%\end{align*}

\subsection{Definición y ejemplos}

\begin{ndef}
  Sea $V(\rm I\!R) $ un espacio vectorial. Diremos que $b$ es \textbf{bilineal} si se verifica:
   \begin{align*}
    b: V \times V & \rightarrow \rm I\!R \\
    (a,b) & \mapsto \varphi(a,b)
  \end{align*}
  \begin{itemize}
  \item[(1.1)] $b(u+u',v)=b(u,v)+b(u',v) \hspace{0.5cm} \forall u, u', v \in V $
  \item[(1.2)] $b(\alpha u,v)=\alpha b(u,v) \hspace{0.5cm} \forall u, u', v \in V,\hspace{0.2cm} \forall \alpha \in \rm I\!R $
  \item[(2.1)] $b(u,v+v')=b(u,v)+b(u,v') \hspace{0.5cm} \forall u, v, v' \in V $
  \item[(2.2)] $b(u,\alpha v)=\alpha b(u,v) \hspace{0.5cm} \forall u, u', v \in V,\hspace{0.2cm} \forall \alpha \in \rm I\!R $
  \end{itemize}
Las cuales se resumen en las siguientes:
  \begin{itemize}
  \item[(1)] $b(\alpha u+\beta u',v)=\alpha b(u,v)+ \beta b(u',v) \hspace{0.5cm} \forall u, u', v \in V,\hspace{0.2cm} \forall \alpha, \beta \in \rm I\!R$
  \item[(2)] $b(u,\alpha v+\beta v')=\alpha b(u,v)+ \beta b(u,v') \hspace{0.5cm} \forall u, v, v' \in V,\hspace{0.2cm} \forall \alpha, \beta \in \rm I\!R$
  \end{itemize}
\end{ndef}

Notaremos $\mathfrak{B}(V)$ al \textbf{conjunto de las formas bilineales} en un espacio vectorial $V$.

\pagebreak

\begin{ejer}
  $(\mathfrak{B}(V), +, \cdot)$ tiene una estructura de espacio vectorial sobre $\rm I\!R$.
\end{ejer}

\begin{sol}
  Basta comprobar que:
  
\end{sol}

\begin{nth}
  Ut sit amet sem id nunc feugiat lacinia sit amet eu felis. Quisque gravida, nisi eget elementum aliquam, arcu urna sodales sapien, sed gravida arcu nisl aliquam lectus. Vestibulum mollis mollis mauris et tristique. Aliquam erat volutpat. Suspendisse in lorem mi.

  $${\frac {d}{dx}}\arctan(\sin({x}^{2}))=-2\,{\frac {\cos({x}^{2})x}{-2+ \left (\cos({x}^{2})\right )^{2}}}$$
\end{nth}

\begin{proof}
  Sed sodales rhoncus lacus non feugiat. Vivamus mi nisl, commodo ut vulputate sed, facilisis at risus. Duis eget cursus mauris. Sed sed augue sit amet enim elementum accumsan. Curabitur imperdiet risus lectus, id volutpat nibh malesuada vitae. Praesent vel libero in justo porta congue et at justo. Cras iaculis eleifend nisl id malesuada. Aenean ac arcu non felis convallis placerat id nec libero. Proin faucibus a ligula et tempor. Nam commodo venenatis ultrices. Nunc tempor hendrerit dolor eget tincidunt. Integer lacinia mi aliquam, faucibus leo finibus, aliquet elit. Mauris laoreet facilisis sagittis. Mauris et varius magna.
\end{proof}

\begin{nprop}
  Suspendisse in tortor sit amet ex feugiat aliquet et vitae nisl. Phasellus auctor imperdiet odio, eget vestibulum augue. Nunc finibus leo rhoncus nisl imperdiet, ac tincidunt nisi fermentum. Phasellus vestibulum ex odio, id laoreet nisl molestie sed.

  $$\frac{d}{dx}\left( \int_{0}^{x} f(u)\,du\right)=f(x)$$
\end{nprop}

\begin{ndef}
  Suspendisse maximus hendrerit dui. Sed ac dapibus enim. Phasellus tempor dolor et metus ullamcorper pretium. Morbi varius ac orci ac volutpat. Pellentesque gravida urna risus, ut porta felis mattis vitae. Suspendisse vulputate sagittis mauris, ac bibendum mauris volutpat iaculis. Suspendisse suscipit ac quam ut commodo. Fusce id leo sollicitudin, placerat velit nec, tempor velit. Donec sed dapibus ex. Duis tincidunt sem non velit blandit sollicitudin eget at sem.
\end{ndef}

\begin{ejemplo}
  Sed dignissim, purus eu consequat volutpat, lectus nulla ullamcorper tellus, tempus blandit sem purus eget ligula. Cras sapien purus, placerat laoreet eros id, tristique imperdiet est. Sed orci purus, hendrerit finibus orci sed, elementum gravida libero. Sed vitae imperdiet magna, nec bibendum velit. Nunc tincidunt risus eget mi pulvinar, nec rhoncus enim dictum.
\end{ejemplo}

In ultricies accumsan faucibus. Quisque faucibus mi vel augue cursus, eu cursus neque pulvinar. Praesent eget velit in nulla interdum efficitur at nec turpis. Sed massa velit\footnotemark, consequat ac dui pharetra, sodales sodales quam. In rhoncus turpis ac elementum imperdiet. Etiam ipsum metus, euismod vel viverra eget, gravida eu felis. Nulla facilisi. Sed eget elementum justo. Morbi fermentum sapien vitae erat fermentum blandit quis ut leo. Morbi aliquam libero eu odio egestas, et condimentum sapien dignissim. Morbi quis lacinia tellus. Sed mattis suscipit feugiat.

\footnotetext{Sed lobortis eu ante nec commodo. Cras ut feugiat mauris. Nullam mollis lacus nisi, eu tristique eros sagittis ac. Nullam mattis tincidunt maximus. Integer quis diam justo. Pellentesque in pharetra nisi. Praesent at interdum dolor. Suspendisse nunc nulla, lobortis vitae libero non, consequat pretium mi. Sed sollicitudin}




Sed lobortis eu ante nec commodo. Cras ut feugiat mauris. Nullam mollis lacus nisi, eu tristique eros sagittis ac. Nullam mattis tincidunt maximus. Integer quis diam justo. Pellentesque in pharetra nisi. Praesent at interdum dolor. Suspendisse nunc nulla, lobortis vitae libero non, consequat pretium mi. Sed sollicitudin, erat at lacinia venenatis, lectus magna tincidunt justo, id condimentum tellus nulla non lectus. Vestibulum sem libero, ultrices vel finibus sit amet, lobortis vitae augue. Nunc sit amet diam egestas, aliquet augue vel, rutrum erat. Nunc scelerisque ultricies nulla, sit amet euismod quam lobortis eu.

Etiam in enim in lectus tempor elementum sed nec arcu. Cras nec nisl non turpis molestie vulputate eu at eros. Nulla facilisis molestie elit eget varius. Aenean vel ex euismod, scelerisque purus nec, porta sem. Pellentesque ullamcorper, augue sit amet fringilla dignissim, nulla justo elementum tellus, id semper metus lacus non diam.

Ut auctor fermentum ligula. In non diam commodo, efficitur enim vel, pretium tortor. Suspendisse mollis elit quis leo vehicula posuere. Integer imperdiet malesuada diam non vestibulum. Aliquam felis tortor, fringilla in faucibus ac, malesuada in metus. Vestibulum ullamcorper egestas nisi vel ultrices. Pellentesque euismod arcu eu nisi congue, a accumsan metus sollicitudin. Aenean ornare cursus feugiat. Interdum et malesuada fames ac ante ipsum primis in faucibus. Phasellus et gravida neque, a tristique ex. Curabitur in eros eu urna suscipit aliquam nec euismod mauris. Nam sollicitudin hendrerit accumsan. Nunc semper lorem risus, at eleifend turpis mollis ac.

Aliquam vitae sem ut arcu tincidunt imperdiet non non nulla. Suspendisse eu maximus lacus. Sed et lorem sapien. Sed id facilisis erat. Nulla porttitor, mauris non lacinia congue, arcu libero blandit quam, sed laoreet magna lectus iaculis justo. Praesent commodo aliquam elementum. Vestibulum volutpat fermentum finibus. Quisque semper vitae mauris id fringilla. Pellentesque habitant morbi tristique senectus et netus et malesuada fames ac turpis egestas. Nunc vulputate imperdiet mollis. Aenean accumsan rhoncus est, sed molestie ligula congue in. Donec aliquet ante tempor pellentesque posuere.

Etiam sit amet congue nibh. Morbi auctor vitae enim ut porttitor. Mauris finibus tellus ligula, at mattis tortor auctor in. Morbi dignissim pretium elit, non pretium massa ultricies at. Ut volutpat efficitur est quis varius. Vestibulum cursus non elit et luctus. Cras elit orci, facilisis non blandit a, vulputate dignissim libero. Aliquam condimentum ut velit quis facilisis. Donec in mi nec dolor mattis placerat vitae vitae turpis. Integer blandit erat et tortor ornare posuere. Donec ornare nisl eget laoreet tincidunt. Donec gravida eros a lorem tincidunt pharetra. Curabitur porta sem at ex consectetur efficitur. Phasellus semper porttitor consequat. Etiam bibendum condimentum ligula ac viverra.

Aliquam scelerisque sit amet arcu a egestas. Quisque malesuada ornare risus, ultrices lacinia lacus congue sit amet. Integer porttitor in dolor nec vehicula. Nulla sagittis odio enim, et congue tellus lacinia a. Nam tempor vulputate varius. Quisque porta hendrerit libero, vel facilisis eros fringilla vitae. Fusce ac felis ut mi placerat volutpat id vitae quam. Curabitur porttitor, eros in malesuada laoreet, velit enim scelerisque mi, a bibendum quam libero id lacus. Praesent malesuada leo ut turpis bibendum, vitae hendrerit metus consectetur. Fusce libero urna, porta at tempus ut, bibendum sit amet lacus. Duis dictum elementum fermentum. Praesent risus odio, tempor sit amet est eu, efficitur egestas massa.

Integer rutrum est eu sodales vehicula. Donec sagittis leo ac augue consequat pellentesque. Maecenas ultrices vehicula augue ac pellentesque. Fusce id convallis orci. Mauris id facilisis lorem. Phasellus eros urna, eleifend vitae dignissim nec, blandit non neque. Etiam id mauris sollicitudin, sollicitudin urna sed, facilisis lectus. Nulla facilisi. Suspendisse eget tristique erat.

Nam aliquet augue quis sapien viverra, scelerisque vestibulum nisl blandit. Duis ut efficitur purus, sed tincidunt purus. Morbi sed nisi tempus purus finibus maximus et a metus. Aenean nisl elit, dignissim et sodales vitae, mollis eget ex. Suspendisse hendrerit tincidunt ex, nec tempus urna ultricies vel. Aenean sed mattis risus. Proin sit amet est porta, posuere ipsum sit amet, tristique leo. Donec sed porta velit, a vestibulum orci. Phasellus pharetra semper lectus sed commodo. Aliquam pellentesque luctus leo. Sed commodo tellus eu mauris suscipit, et lobortis arcu tincidunt. Sed feugiat magna quis ligula viverra, nec pellentesque enim consequat.

Sed eleifend malesuada augue, eget dapibus ante scelerisque eget. Vestibulum gravida dui eu congue pulvinar. Sed sed purus in nisi molestie facilisis eget at nisi. Interdum et malesuada fames ac ante ipsum primis in faucibus. Vivamus a laoreet sem, nec auctor tellus. Donec sed pharetra nisl. Pellentesque accumsan quam a semper dapibus. Proin elementum viverra metus sed ultrices.

Aenean finibus ex at magna bibendum, ut vehicula dolor tempor. In semper, ipsum suscipit tincidunt posuere, diam arcu accumsan ligula, mollis hendrerit massa ipsum et enim. Duis finibus, urna ac interdum luctus, lorem enim faucibus dolor, a sollicitudin dui augue vel nibh. Nulla sed efficitur est. Mauris ut metus tincidunt, iaculis turpis sed, mattis lacus. Vestibulum ut purus maximus massa tristique ultrices sit amet ut tortor. Pellentesque habitant morbi tristique senectus et netus et malesuada fames ac turpis egestas.


\end{document}
