%%%%%%%%%%%%%%%%%%%%%%%%%%%%%%%%%%%%%%%%%%%%%%%%%%%%%%%%%%%%%%%%%%%%%%%%%%
%
% Plantilla para libro de texto de matemáticas.
%
% Esta plantilla ha sido desarrollada desde cero, pero utiliza algunas partes
% del código de la plantilla original utilizada en apuntesDGIIM
% (https://github.com/libreim/apuntesDGIIM), basada a su vez en las plantillas
% 'Short Sectioned Assignment' de Frits Wenneker (http://www.howtotex.com),
% 'Plantilla de Trabajo' de Mario Román y 'Plantilla básica de Latex en Español'
% de Andrés Herrera Poyatos (https://github.com/andreshp). También recoge
% ideas de la plantilla 'Multi-Purpose Large Font Title Page' de
% Frits Wenneker y Vel (vel@latextemplates.com).
%
% Licencia:
% CC BY-NC-SA 4.0 (https://creativecommons.org/licenses/by-nc-sa/4.0/)
%
%%%%%%%%%%%%%%%%%%%%%%%%%%%%%%%%%%%%%%%%%%%%%%%%%%%%%%%%%%%%%%%%%%%%%%%%%

% ---------------------------------------------------------------------------
% CONFIGURACIÓN BÁSICA DEL DOCUMENTO
% ---------------------------------------------------------------------------

%\documentclass[11pt, a4paper, twoside]{article} % Usar para imprimir
\documentclass[10pt, a4paper]{article}

\linespread{1.3}            % Espaciado entre líneas.
\setlength\parindent{0pt}   % No indentar el texto por defecto.
\setlength\parskip{7pt}

% ---------------------------------------------------------------------------
% PAQUETES BÁSICOS
% ---------------------------------------------------------------------------

% IDIOMA
\usepackage[utf8]{inputenc}
\usepackage[spanish, es-tabla, es-lcroman, es-noquoting]{babel}

% MATEMÁTICAS
\usepackage{amsmath}    % Paquete básico de matemáticas
\usepackage{amsthm}     % Teoremas
\usepackage{mathrsfs}   % Fuente para ciertas letras utilizadas en matemáticas

% FUENTES
\usepackage{newpxtext, newpxmath}   % Fuente similar a Palatino
\usepackage{FiraSans}                 % Fuente sans serif
\usepackage[T1]{fontenc}
\usepackage[italic]{mathastext}     % Utiliza la fuente del documento
                                    % en los entornos matemáticos

% MÁRGENES
\usepackage[margin=2.5cm, top=3cm]{geometry}

% LISTAS
\usepackage{enumitem}       % Mejores listas
\setlist{leftmargin=.5in}   % Especifica la indentación para las listas.

% Listas ordenadas con números romanos (i), (ii), etc.
\newenvironment{nlist}
{\begin{enumerate}
    \renewcommand\labelenumi{(\emph{\roman{enumi})}}}
  {\end{enumerate}}

%  OTROS
\usepackage{hyperref}   % Enlaces
\usepackage{graphicx}   % Permite incluir gráficos en el documento
\usepackage{relsize}
 
 
% ---------------------------------------------------------------------------
% COMANDOS PERSONALIZADOS
% ---------------------------------------------------------------------------

% \equalto
\newcommand{\verteq}{\rotatebox{90}{$\,=$}}
\newcommand{\equalto}[2]{\underset{\scriptstyle\overset{\mkern4mu\verteq}{#2}}{#1}}


% ---------------------------------------------------------------------------
% COLORES
% ---------------------------------------------------------------------------

\usepackage{xcolor}     % Permite definir y utilizar colores

\definecolor{50}{HTML}{E0F2F1}
\definecolor{100}{HTML}{B2DFDB}
\definecolor{200}{HTML}{80CBC4}
\definecolor{300}{HTML}{4DB6AC}
\definecolor{400}{HTML}{26A69A}
\definecolor{500}{HTML}{009688}
\definecolor{600}{HTML}{00897B}
\definecolor{700}{HTML}{00796B}
\definecolor{800}{HTML}{00695C}
\definecolor{900}{HTML}{004D40}

% ---------------------------------------------------------------------------
% DISEÑO DE PÁGINA
% ---------------------------------------------------------------------------

\usepackage{pagecolor}
\usepackage{afterpage}

% ---------------------------------------------------------------------------
% CABECERA Y PIE DE PÁGINA
% ---------------------------------------------------------------------------

\usepackage{fancyhdr}   % Paquete para cabeceras y pies de página

% Indica que las páginas usarán la configuración de fancyhdr
\pagestyle{fancy}
\fancyhf{}

% Representa la sección de la cabecera
\renewcommand{\sectionmark}[1]{%
\markboth{#1}{}}

% Representa la subsección de la cabecera
\renewcommand{\subsectionmark}[1]{%
\markright{#1}{}}

% Parte derecha de la cabecera
\fancyhead[LE,RO]{\sffamily\textsl{\rightmark} \hspace{1em}  \textcolor{500}{\rule[-0.4ex]{0.2ex}{1.2em}} \hspace{1em} \thepage}

% Parte izquierda de la cabecera
\fancyhead[RE,LO]{\sffamily{\leftmark}}

% Elimina la línea de la cabecera
\renewcommand{\headrulewidth}{0pt}

% Controla la altura de la cabecera para que no haya errores
\setlength{\headheight}{14pt}

% ---------------------------------------------------------------------------
% TÍTULOS DE PARTES Y SECCIONES
% ---------------------------------------------------------------------------

\usepackage{titlesec}

% Estilo de los títulos de las partes
\titleformat{\part}[hang]{\Huge\bfseries\sffamily}{\thepart\hspace{20pt}\textcolor{500}{|}\hspace{20pt}}{0pt}{\Huge\bfseries}
\titlespacing*{\part}{0cm}{-2em}{2em}[0pt]

% Reiniciamos el contador de secciones entre partes (opcional)
\makeatletter
\@addtoreset{section}{part}
\makeatother

% Estilo de los títulos de las secciones, subsecciones y subsubsecciones
\titleformat{\section}
  {\Large\bfseries\sffamily}{\thesection}{1em}{}

\titleformat{\subsection}
  {\Large\sffamily}{\thesubsection}{1em}{}[\vspace{.5em}]

\titleformat{\subsubsection}
  {\sffamily}{\thesubsubsection}{1em}{}

% ---------------------------------------------------------------------------
% ENTORNOS PERSONALIZADOS
% ---------------------------------------------------------------------------

\usepackage{mdframed}

%% DEFINICIONES DE LOS ESTILOS

% Nuevo estilo para definiciones
\newtheoremstyle{definition-style}  % Nombre del estilo
{}                                  % Espacio por encima
{}                                  % Espacio por debajo
{}                                  % Fuente del cuerpo
{}                                  % Identación
{\bf\sffamily}                      % Fuente para la cabecera
{.}                                 % Puntuación tras la cabecera
{.5em}                              % Espacio tras la cabecera
{\thmname{#1}\thmnumber{ #2}\thmnote{ (#3)}}  % Especificación de la cabecera

% Nuevo estilo para notas
\newtheoremstyle{remark-style}
{10pt}
{10pt}
{}
{}
{\itshape \sffamily}
{.}
{.5em}
{}

% Nuevo estilo para teoremas y proposiciones
\newtheoremstyle{theorem-style}
{}
{}
{}
{}
{\bfseries \sffamily}
{.}
{.5em}
{\thmname{#1}\thmnumber{ #2}\thmnote{ (#3)}}

% Nuevo estilo para ejemplos
\newtheoremstyle{example-style}
{10pt}
{10pt}
{}
{}
{\bf \sffamily}
{}
{.5em}
{\thmname{#1}\thmnumber{ #2.}\thmnote{ #3.}}

% Nuevo estilo para la demostración

\makeatletter
\renewenvironment{proof}[1][\proofname] {\par\pushQED{\qed}\normalfont\topsep6\p@\@plus6\p@\relax\trivlist\item[\hskip\labelsep\itshape\sffamily#1\@addpunct{.}]\ignorespaces}{\popQED\endtrivlist\@endpefalse}
\makeatother

%% ASIGNACIÓN DE LOS ESTILOS

% Teoremas, proposiciones y corolarios
\theoremstyle{theorem-style}
\newtheorem{nth}{Teorema}[section]
\newtheorem{nprop}{Proposición}[section]
\newtheorem{ncor}{Corolario}[section]
\newtheorem{lema}{Lema}[section]

% Definiciones
\theoremstyle{definition-style}
\newtheorem{ndef}{Definición}[section]

% Notas
\theoremstyle{remark-style}
\newtheorem*{nota}{Nota}

% Ejemplos
\theoremstyle{example-style}
\newtheorem{ejemplo}{Ejemplo}[section]

% Ejercicios y solución
\theoremstyle{definition-style}
\newtheorem{ejer}{Ejercicio}[section]

\theoremstyle{remark-style}
\newtheorem*{sol}{Solución}

%% MARCOS DE LOS ESTILOS

% Configuración general de mdframe, los estilos de los teoremas, etc
\mdfsetup{
  skipabove=1em,
  skipbelow=1em,
  innertopmargin=1em,
  innerbottommargin=1em,
  splittopskip=2\topsep,
}

% Definimos los marcos de los estilos

\mdfdefinestyle{nth-frame}{
	linewidth=2pt, %
	linecolor= 500, %
	topline=false, %
	bottomline=false, %
	rightline=false,%
	leftmargin=0em, %
	innerleftmargin=1em, %
  innerrightmargin=1em,
	rightmargin=0em, %
}%

\mdfdefinestyle{nprop-frame}{
	linewidth=2pt, %
	linecolor= 300, %
	topline=false, %
	bottomline=false, %
	rightline=false,%
	leftmargin=0pt, %
	innerleftmargin=1em, %
	innerrightmargin=1em,
	rightmargin=0pt, %
}%

\mdfdefinestyle{ndef-frame}{
	linewidth=2pt, %
	linecolor= 500, %
	backgroundcolor= 50,
	topline=false, %
	bottomline=false, %
	rightline=false,%
	leftmargin=0pt, %
	innerleftmargin=1em, %
	innerrightmargin=1em,
	rightmargin=0pt, %
}%

\mdfdefinestyle{ejer-frame}{
	linewidth=2pt, %
	linecolor= 300, %
	backgroundcolor= 50,
	topline=false, %
	bottomline=false, %
	rightline=false,%
	leftmargin=0pt, %
	innerleftmargin=1em, %
	innerrightmargin=1em,
	rightmargin=0pt, %
}%

\mdfdefinestyle{ejemplo-frame}{
	linewidth=0pt, %
	linecolor= 300, %
	leftline=false, %
	rightline=false, %
	leftmargin=0pt, %
	innerleftmargin=1.3em, %
	innerrightmargin=1em,
	rightmargin=0pt, %
	innertopmargin=0em,%
	innerbottommargin=0em, %
	splittopskip=\topskip, %
}%

% Asignamos los marcos a los estilos
\surroundwithmdframed[style=nth-frame]{nth}
\surroundwithmdframed[style=nprop-frame]{nprop}
\surroundwithmdframed[style=nprop-frame]{ncor}
\surroundwithmdframed[style=ndef-frame]{ndef}
\surroundwithmdframed[style=ejer-frame]{ejer}
\surroundwithmdframed[style=ejemplo-frame]{ejemplo}
\surroundwithmdframed[style=ejemplo-frame]{sol}

% ---------------------------------------------------------------------------
% CONFIGURACIÓN DE LA PORTADA
% ---------------------------------------------------------------------------

\newcommand{\asignatura}{
  Geometría II
}

\newcommand{\autor}{DGIIMUnderground · LibreIM}

\newcommand{\grado}{1º Doble Grado en Ingeniería Informática y Matemáticas}

\newcommand{\universidad}{Universidad de Granada}

\newcommand{\enlaceweb}{github.com/DGIIMUnderground}

% ---------------------------------------------------------------------------
% CONFIGURACIÓN PERSONALIZADA
% ---------------------------------------------------------------------------

%%%%%%%%%%%%%%%%%%%%%%%%%%%%%%%%%%%%%%%%%%%%%%%%%%%%%%%%%%%%%%%%%%%%%%%%%%%%%
% ---------------------------------------------------------------------------
% COMIENZO DEL DOCUMENTO
% ---------------------------------------------------------------------------
%%%%%%%%%%%%%%%%%%%%%%%%%%%%%%%%%%%%%%%%%%%%%%%%%%%%%%%%%%%%%%%%%%%%%%%%%%%%%

\begin{document}

% ---------------------------------------------------------------------------
% PORTADA EXTERIOR
% ---------------------------------------------------------------------------

\newpagecolor{500}\afterpage{\restorepagecolor} % Color de la página
\begin{titlepage}

  % Título del documento
	\parbox[t]{\textwidth}{
			\raggedright % Texto alineado a la izquierda
			\fontsize{50pt}{50pt}\selectfont\sffamily\color{white}{
			  \textbf{\asignatura}
      }
	}

	\vfill

	%% Autor e información del documento
	\parbox[t]{\textwidth}{
		\raggedright % Texto alineado a la izquierda
		\sffamily\large\color{white}
		{\Large \autor }\\[4pt]
		\grado\\
		\universidad\\[4pt]
		\texttt{\enlaceweb}
	}

\end{titlepage}

% ---------------------------------------------------------------------------
% PÁGINA DE LICENCIA
% ---------------------------------------------------------------------------

\thispagestyle{empty}
\null
\vfill

%% Información sobre la licencia
\parbox[t]{\textwidth}{
  \includegraphics{by-nc-sa.pdf}\\[4pt]
  \raggedright % Texto alineado a la izquierda
  \sffamily\large
  {\Large Este libro se distribuye bajo una licencia CC BY-NC-SA 4.0.}\\[4pt]
  Eres libre de distribuir y adaptar el material siempre que reconozcas a los\\
  autores originales del documento, no lo utilices para fines comerciales\\
  y lo distribuyas bajo la misma licencia.\\[4pt]
  \texttt{creativecommons.org/licenses/by-nc-sa/4.0/}
}

% ---------------------------------------------------------------------------
% PORTADA INTERIOR
% ---------------------------------------------------------------------------

\begin{titlepage}

  % Título del documento
	\parbox[t]{\textwidth}{
			\raggedright % Texto alineado a la izquierda
			\fontsize{50pt}{50pt}\selectfont\sffamily\color{500}{
			  \textbf{\asignatura}
      }
	}

	\vfill

	%% Autor e información del documento
	\parbox[t]{\textwidth}{
		\raggedright % Texto alineado a la izquierda
		\sffamily\large
		{\Large \autor}\\[4pt]
		\grado\\
		\universidad\\[4pt]
		\texttt{\enlaceweb}
	}

\end{titlepage}

% ---------------------------------------------------------------------------
% ÍNDICE
% ---------------------------------------------------------------------------

\thispagestyle{empty}
\tableofcontents
\newpage

% ---------------------------------------------------------------------------
% CONTENIDO
% ---------------------------------------------------------------------------

\part{Teoría}

\section*{Introducción}

Lorem ipsum dolor sit amet, consectetur adipiscing elit. Ut a nisi id mi dapibus commodo. Nam ac libero ultrices, posuere erat eu, tempor dolor. Curabitur porttitor, nulla at consequat mollis, mi turpis varius elit, vitae eleifend turpis est vel sem. Sed ut vehicula quam. Praesent id sem sed sapien tincidunt iaculis at quis dolor. Integer et magna quis sapien elementum pharetra. Pellentesque porttitor dapibus nibh, eget ullamcorper risus eleifend vitae. Nulla ac nulla nec orci scelerisque eleifend id sit amet arcu. Vivamus sodales, nibh in aliquet fringilla, quam erat tristique diam, quis dignissim justo mauris et lorem.

\pagebreak

\section{Formas bilineales y formas cuadráticas}

Como introducción, tomaremos el \textbf{producto escalar}:

\hspace{1cm} $ \rm I\!R^n \text{ espacio vectoral sobre un cuerpo } K, \hspace{0.5cm} x=(x_1,\ldots,x_n), y=(y_1,\ldots,y_n) \in \rm I\!R^n$ \\
$$\cdot \hspace{2mm} : \hspace{2mm} \rm I\!R^n \hspace{2mm} \rightarrow \hspace{2mm} \rm I\!R, \hspace{0.5cm} (x,y) \hspace{2mm} \mapsto \hspace{2mm} x \cdot y = x_1y_1+\ldots+x_ny_n=\sum^n_{i=1}x_iy_i$$
\subsubsection*{Propiedades del producto escalar}

\begin{enumerate}
	\item $ (x+x')\cdot y=x \cdot y+x' \cdot y \text{\hspace{0.5cm}y\hspace{0.5cm}} x\cdot(y+y') = x \cdot y + x\cdot y' \hspace{0.5cm} \forall x, y, x', y' \in \rm I\!R^n$
	\item $ (\alpha x)\cdot y = \alpha x\cdot y \text{\hspace{0.5cm}y\hspace{0.5cm}} x\cdot(\alpha y) = \alpha x\cdot y \hspace{0.5cm} \forall x, y \in \rm I\!R ^n, \hspace{0.2cm} \forall \alpha \in K $
	\item $ x\cdot y = y\cdot x \hspace{0.5cm} \forall x, y \in \rm I\!R ^n $
	\item $ x\cdot x \geq 0 \text{\hspace{0.5cm}y\hspace{0.5cm}} x\cdot x = 0 \iff x=0 \hspace{0.5cm} \forall x \in \rm I\!R ^n$
\end{enumerate}

%\begin{align*}
%          \sin A \cos B &= \frac{1}{2}\left[ \sin(A-B)+\sin(A+B) \right] \\
%         \sin A \sin B &= \frac{1}{2}\left[ \sin(A-B)-\cos(A+B) \right] \\
%         \cos A \cos B &= \frac{1}{2}\left[ \cos(A-B)+\cos(A+B) \right] \\
%\end{align*}

\subsection{Definición y ejemplos}

\begin{ndef}[Forma bilineal]
  Sea $V(\rm I\!R) $ un espacio vectorial. Diremos que la aplicación $b$ es \textbf{bilineal} si se verifica lo siguiente:
   \begin{align*}
    b \hspace{2mm} : \hspace{2mm} V \times V \hspace{2mm} &\rightarrow \hspace{2mm} \rm I\!R \\
    (a,b) \hspace{2mm} & \mapsto \hspace{2mm} \varphi(a,b)
  \end{align*}
  \begin{itemize}
  \item[(1.1)] $b(u+u',v)=b(u,v)+b(u',v) \hspace{0.5cm} \forall u, u', v \in V $
  \item[(1.2)] $b(\alpha u,v)=\alpha b(u,v) \hspace{0.5cm} \forall u, u', v \in V,\hspace{0.2cm} \forall \alpha \in \rm I\!R $
  \item[(2.1)] $b(u,v+v')=b(u,v)+b(u,v') \hspace{0.5cm} \forall u, v, v' \in V $
  \item[(2.2)] $b(u,\alpha v)=\alpha b(u,v) \hspace{0.5cm} \forall u, u', v \in V,\hspace{0.2cm} \forall \alpha \in \rm I\!R $
  \end{itemize}
Las cuales se resumen en las siguientes:
  \begin{itemize}
  \item[(1)] $b(\alpha u+\beta u',v)=\alpha b(u,v)+ \beta b(u',v) \hspace{0.5cm} \forall u, u', v \in V,\hspace{0.2cm} \forall \alpha, \beta \in \rm I\!R$
  \item[(2)] $b(u,\alpha v+\beta v')=\alpha b(u,v)+ \beta b(u,v') \hspace{0.5cm} \forall u, v, v' \in V,\hspace{0.2cm} \forall \alpha, \beta \in \rm I\!R$
  \end{itemize}
  Es decir, que $b$ es lineal por cada una de las dos variables.
\end{ndef}

Notaremos $\mathfrak{B}(V)$ al \textbf{conjunto de las formas bilineales} en un espacio vectorial $V$.

\pagebreak

\begin{ejer}
  $(\mathfrak{B}(V), +, \cdot)$ tiene una estructura de espacio vectorial sobre $\rm I\!R$.
\end{ejer}

\begin{sol}
  Basta comprobar que:
  \begin{align*}
	(b_1+b_2)(u,v)&=b_1(u,v)+b_2(u,v)\\(\alpha b)(u,v) &= \alpha b(u,v)
\end{align*}
\end{sol}

\subsubsection{Propiedades}

\begin{enumerate}
	\item $b(0,v) = b(u, 0) = 0 \hspace{5mm}\forall u,v \in V$
	\begin{proof}
	$b(0,v)=b(0\cdot v, v) = 0 \cdot b(v,v) = 0 $
\end{proof}
	\item $b(-u,v)=-b(u,v); \hspace{2mm} b(u,-v)=-b(u,v) \hspace{5mm} \forall u,v \in V$
	\begin{proof}
	$b(-u,v)=b((-1)u,v)=(-1)b(u,v)$
	\end{proof}
	\item $ \mathlarger{b\left(\sum^n_{i=1}\alpha_iu_i, \sum_{j=1}^m \beta_j v_j \right) = \sum_{i=1}^n \sum_{j=1}^m \alpha_i \beta_j b(u_i,v_j)}$
	\begin{proof}
	Procederemos por inducción.
	
	\boxed{\emph{n}=1, \emph{m}=1} \hspace{3mm} $b(\alpha_1 u_1, \beta_1 v_1) = \alpha_1 \beta_1 b(u_1,v_1)$
	
	Supuesto cierto para $n-1$, $m-1$, ¿es cierto para $n$, $m$?
	
	\boxed{\emph{n}, \emph{m}}
	\begin{align*}
	b \left( \sum_{i=1}^{n-1} \alpha_i u_i + \alpha_n u_n, \sum_{j=1}^{m-1} + \beta_m v_m \right) \stackrel{(1)}{=}
	b \left( \sum_{i=1}^{n-1} \alpha_i u_i, \sum_{j=1}^{m-1} + \beta_m v_m \right) +
	\alpha_n b \left(u_n, \sum_{j=1}^{m-1} + \beta_m v_m \right) \\
	\stackrel{(1)}{=}b \left( \sum_{i=1}^{n-1} \alpha_i u_i, \sum_{j=1}^{m-1} + \beta_m v_m \right) +
	\beta_m b \left( \sum_{i=1}^{n-1} \alpha_i u_i,  v_m \right) +
	\alpha_n b \left( u_n, \sum_{j=1}^{m-1} + \beta_m v_m \right) +
	\alpha_n \beta_m b \left( u_n, v_m \right) \\
	\stackrel{(2)}{=} \sum_{i=1}^{n-1} \sum_{j=1}^{m-1} (u_i, v_j) + \beta_m \sum_{i=1}^{n-1} \alpha_i (u_i,v_m) + \alpha_n  \sum_{j=1}^{m-1}(u_n, v_j) + \alpha_n \beta_m b(u_n, v_n)=\sum_{i=1}^{n} \sum_{j=1}^{m}	\alpha_i \beta_j b(u_i, v_j)
\end{align*}
Donde en (1) hemos aplicado que $b$ es bilineal, y en (2) la hipótesis de inducción.
\end{proof}
\end{enumerate}

\pagebreak

\subsubsection{Ejemplos de formas bilineales}

En los siguientes ejemplos, a menos que se especifique lo contrario, tenemos un espacio vectorial $V(\rm I\!R)$, una base $\mathcal{B}$ de $V$ y $A \in M_n(\rm I\!R)$. Podemos definir entonces una forma bilineal $b \hspace{1mm} : \hspace{1mm} V\times V \hspace{1mm} \rightarrow \hspace{1mm} R$ como:

\begin{enumerate}
	\item $V=\rm I\!R^3, \hspace{2mm} \mathcal{B} = \{(1,0,0),(0,1,0),(0,0,1)\}$
	
	$M(b_0, \mathcal{B}) = \left( \begin{matrix} 1&0&0\\0&1&0\\0&0&1 \end{matrix} \right) = I_3$ \\
	
	Podemos generalizar: en $V = \rm I\!R^n$ consideraremos $b_0$ como la forma bilineal asociada a $I_n$ si $\mathcal{B} = \mathcal{B}_u$, en cuyo caso $M(b_0, \mathcal{B}_u) = I_n$.
	
\end{enumerate}

\pagebreak

\subsection{Matriz asociada a una forma bilineal}
Sea $V(\rm I\!R)$ un espacio vectorial real, $\mathcal{B} = \{u_1, \ldots, u_n\}$ base de $V$. Dado $b \in \mathfrak{B}(V)$, tenemos entonces que:
$$ u,v \in V \hspace{1cm} u=\sum_{i=1}^n x_i u_i\hspace{0.5mm}; \hspace{2mm} v = \sum_{j=1}^n y_ju_j \hspace{1cm}
	x = \left( \begin{matrix} x_1 \\ \vdots \\ x_n \end{matrix} \right)\hspace{0.5mm}; \hspace{2mm}
	y = \left( \begin{matrix} y_1 \\ \vdots \\ y_n \end{matrix} \right)$$
Luego $ \mathlarger{b(u,v)=b\left(\sum_{i=1}^n x_iu_i, \sum_{j=1}^n y_ju_j\right) = \sum_{i=1}^n \sum_{j=1}^m x_i y_j b(u_i,u_j)}$.

A la matriz cuya entrada $(i,j)$ es $b(u_i,u_j)$ la denotaremos $M(b, \mathcal{B})$ y la llamaremos \textbf{matriz de la forma bilineal $b$ en la base $\mathcal{B}$}. De este modo:
\begin{align*}
	b(u,v) &= \left(\begin{matrix} x_1 & \cdots & x_n \end{matrix} \right) \left( \begin{matrix} b(u_1, u_1) & \ldots & b(u_1, u_n) \\ \vdots &\ddots & \vdots \\ b(u_n,u_1) & \cdots & b(u_n,u_n) \end{matrix} \right)
	\left( \begin{matrix} y_1 \\ \vdots \\ y_n \end{matrix} \right) = x^tM(b, \mathcal{B})y
\end{align*}

\begin{nota}
	Si hacemos $M(b, \mathcal{B})\left( \begin{matrix} y_1 \\ \vdots \\ y_n \end{matrix} \right)$ obtendremos en la fila $i$-ésima: $y_1b(u_i,u_1))+y_2b(u_i,u_2)+\ldots+y_nb(u_i,u_n)$.
	Es por tanto claro que $\left( \begin{matrix} x_1 & \cdots & x_n \end{matrix} \right)M(b, \mathcal{B})\left( \begin{matrix} y_1 \\ \vdots \\ y_n \end{matrix} \right)=b(u,v)$.
\end{nota}
	
\vspace{2mm}

Dada $\mathcal{B}$ base de $V$ podemos definir:
\begin{align*}
	F_\mathcal{B} \hspace{2mm} : \hspace{2mm} \mathfrak{B}(V) \hspace{2mm} &\rightarrow \hspace{2mm} M_n(\rm I\!R) \\
	b \hspace{2mm} &\mapsto \hspace{2mm} M(b, \mathfrak{B})
\end{align*}

Tenemos un isomorfismo de espacios vectoriales, con $\text{dim}(\mathfrak{B}(V))=n^2$.
\begin{proof}
$$ \begin{matrix} F_\mathcal{B}(\lambda b + \mu b') & = & \lambda F_\mathcal{B}(b) + \mu F_\mathcal{B}(b') & \\ \rotatebox{90}{$\,=$} & & \rotatebox{90}{$\,=$} & \hspace{1cm}\forall b,b' \in \mathfrak{B}(V), \hspace{2mm} \forall \lambda, \mu \in \rm I\!R \\ \equalto{M(\lambda b + \mu b', \mathcal{B})}{(a_{ij})} & = & \lambda \equalto{M(b, \mathcal{B})}{(b_{ij})} + \mu \equalto{M(b', \mathcal{B})}{(c_{ij})} \end{matrix} $$

De hecho estas igualdades son así debido a que se verifica:
$$a_{ij}=(\lambda b + \mu b' )(u_i, u_j)=\lambda b(u_i,u_j) + \mu b'(u_i,u_j)$$
$$ b_{ij} = b(u_i,u_j)\hspace{1cm} c_{ij} = b'(u_i,u_j)$$
\end{proof}

Es más, podemos ver que $F_\mathcal{B}$ es inyectiva:
$$ \text{Ker}(F_\mathcal{B})=\{b \in \mathfrak{B}(V) : F(b) = 0 \} = \{b \in \mathfrak{B}(V) : M(b, \mathcal{B}) = 0 \} = \{0\}$$

\end{document}
